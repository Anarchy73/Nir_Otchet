\section{Фрагменты кода программы <<Матричный калькулятор>>}
\label{app:matrix}

Функция установки размера матрицы B
\begin{minted}[style=bw,
 linenos=true,
 breaklines=true,
 numbersep=5pt,
 tabsize=2,
 fontsize=\small,
 bgcolor=white]{cpp}
private: System::Void set_size_B_Click(System::Object^ sender, System::EventArgs^ e) {
	this->errorProvider1->Clear();
	int n, m;
	bool resn = Int32::TryParse(this->set_size_B_lines->Text, n);
	bool resm = Int32::TryParse(this->set_size_B_columns->Text, m);
	if (!(resn && resm)) {
		this->errorProvider1->SetError(this->input_B, "Неправильный формат ввода размера таблицы");
		return;
	}
	else if (n <= 0 || m <= 0) {
		this->errorProvider1->SetError(this->input_B, "Нельзя задать размер с 0");
		return;
	}


	int line_count = this->input_B->RowCount;

	for (int i = 0; i < line_count; i++) {
		this->input_B->Rows->Remove(this->input_B->Rows[0]);
	}
	int column_count = this->input_B->ColumnCount;
	for (int i = 0; i < column_count; i++) {
		this->input_B->Columns->Remove(this->input_B->Columns[0]);
	}


	for (int i = 0; i < m; i++) {
		this->input_B->Columns->Add("", "");
	}

	this->input_B->Rows->Add(n);
}
\end{minted}
