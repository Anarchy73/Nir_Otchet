\section{Фрагменты кода программы <<Обработка табличных данных. Часть 1>>}
\label{app:table_data}

Функция обработки нажатия на кнопку "Минимум"
\begin{minted}[style=bw,
 linenos=true,
 breaklines=true,
 numbersep=5pt,
 tabsize=2,
 fontsize=\small,
 bgcolor=white]{cpp}
private: System::Void min_btn_Click(System::Object^ sender, System::EventArgs^ e) {
	long long y, maxel = 1e9;
	bool found_chet = false;
	for (int i = 0; i < this->mas_grid->RowCount; i++) {
		bool result = Int64::TryParse(this->mas_grid->Rows[i]->Cells[0]->Value->ToString(), y);
		if (y % 2 == 0) {
			found_chet = true;
			maxel = (y < maxel) ? y : maxel;
		}
	}
	if (found_chet)	this->chet_box->Text = System::Convert::ToString(maxel);
	else {
		this->chet_box->Text = "Четных чисел нет";
	}
}
\end{minted}
Функция добавления числа в таблицу
\begin{minted}[style=bw,
 linenos=true,
 breaklines=true,
 numbersep=5pt,
 tabsize=2,
 fontsize=\small,
 bgcolor=white]{cpp}
private: System::Void mas_add_Click(System::Object^ sender, System::EventArgs^ e) {
	long long x;
	bool result = Int64::TryParse(this->x_input->Text, x);
	if (!result) {
		this->errorProvider1->SetError(this->mas_add, "Ошибка формата ввода");
		return;
	}
	this->mas_grid->Rows->Add(1);
	this->mas_grid->Rows[this->mas_grid->RowCount - 1]->SetValues(x);

}
\end{minted}