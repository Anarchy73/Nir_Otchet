\section{Фрагменты кода программы <<Простые вычисления>>}
\label{app:simple_calculations}

Функция, при нажатии на кнопку <<Вычислить>>.
\begin{minted}[style=bw,
 linenos=true,
 breaklines=true,
 numbersep=5pt,
 tabsize=2,
 fontsize=\small,
 bgcolor=white]{cpp}
private: System::Void solvebtn_Click(System::Object^ sender, System::EventArgs^ e) {
	this->output->Text = ""; // Стираем поле вывода

	errors->SetError(x_input, String::Empty); // обнуляем ошибки
	errors->SetError(y_input, String::Empty);

	Int64 x, y; // Переменные для считывания полей ввода
	double result; // переменная для записи результата

	bool result_x = Int64::TryParse(this->x_input->Text, x); // записываем из полей ввода в соответствующие переменные
	bool result_y = Int64::TryParse(this->y_input->Text, y); // и проверяем на успешность выполнения парсинга

	if (!result_x) { // если неудачно
		errors->SetError(x_input, "Введено не целое число");
	}
	if (!result_y) {
		errors->SetError(y_input, "Введено не целое число");
	}
	if (result_x && result_y) { // если удачно
		if (x * x - y == 0) { // проверка на ОДЗ
			this->output->Text = "Деление на ноль";
		}
		else {
			result = (1.0) * ((x + y) * (x + y) - x * x * x) / (std::abs(x * x - y)); //Считаем
			this->output->Text = System::Convert::ToString(result); //Записываем результат
		}
	}
}
\end{minted}
