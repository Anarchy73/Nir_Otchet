\section{Фрагменты кода программы <<Рекурсивные вычисления>>}
\label{app:recursive_calculations}


Функция обработки нажатия на кнопку <<Вычислить>>.
\begin{minted}[style=bw,
 linenos=true,
 breaklines=true,
 numbersep=5pt,
 tabsize=2,
 fontsize=\small,
 bgcolor=white]{cpp}
private: System::Void Solve_Click(System::Object^ sender, System::EventArgs^ e) {
	ClearAll();
	long long n, m, A;
	bool result_n = Int64::TryParse(this->n_input->Text, n);
	bool result_m = Int64::TryParse(this->m_input->Text, m);
	if (!result_n) {
		errorProvider1->SetError(this->n_input, "Неправильно введено число n.");
	}
	if (!result_m) {
		errorProvider1->SetError(this->m_input, "Неправильно введено число m.");
	}
	if (result_m && result_n) {

		if (n > m) {
			this->Output->Text = "Ошибка: n > m";
		}
		else {
			if (n < 1) {
				this->Output->Text = "" + "Ошибка: n < 1 "+ System::Environment::NewLine;
			}
			if (m < 1) {
				this->Output->Text += "Ошибка: m < 1\n";
			}
			if (n >= 1 && m >= 1) {
				this->Output->Text = "Количество различных размещений из n в m равно: " + System::Convert::ToString(per(n, m));
			}
		}
	}
}
\end{minted}