\section{Фрагменты кода программы <<Использование коллекций в Windows Forms>>}
\label{app:collections}

Функция нахождения максимального нечетного элемента стека.
\begin{minted}[style=bw,
 linenos=true,
 breaklines=true,
 numbersep=5pt,
 tabsize=2,
 fontsize=\small,
 bgcolor=white]{cpp}
private: System::Void max_nech_btn_Click(System::Object^ sender, System::EventArgs^ e) {
	System::Collections::Generic::Stack<int> buf; //вспомогательный стек
	int maxel = -1e9;
	bool nech = false;
	while (s.Count) {//пока стек не пуст
		if (maxel < s.Peek() && s.Peek() % 2 != 0) {
			maxel = s.Peek();
			nech = true;
		} 
		buf.Push(s.Peek()); //записываем во вспомогательный стек первый элемент
		s.Pop(); //удаляем первый элемент из стека


	}
	while (buf.Count) { //пока вспомогательный стек не пуст
		s.Push(buf.Peek()); //записываем в основный стек первый элемент вспомогательного
		buf.Pop(); //удаляем из стека первый элемент
	}
	if (nech) this->max_nech->Text = System::Convert::ToString(maxel); //записываем результат в строку
	else this->max_nech->Text = "Стек не содержит четных элементов";
}
\end{minted}

