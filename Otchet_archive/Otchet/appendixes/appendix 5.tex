\section{Фрагменты кода программы <<Обработка табличных данных. Часть 2>>}
\label{app:table_data_2}

Функция установки размера таблицы.
\begin{minted}[style=bw,
 linenos=true,
 breaklines=true,
 numbersep=5pt,
 tabsize=2,
 fontsize=\small,
 bgcolor=white]{cpp}
private: System::Void set_size_Click(System::Object^ sender, System::EventArgs^ e) {
	this->errorProvider1->Clear();
	int n, m;
	bool resn = Int32::TryParse(this->set_size_lines->Text, n);
	bool resm = Int32::TryParse(this->set_size_columns->Text, m);
	if (!(resn && resm)) {
		this->errorProvider1->SetError(this->input_grid, "Неправильный формат ввода размера таблицы");
		return;
	}
	else if (n == 0 || m == 0) {
		this->errorProvider1->SetError(this->input_grid, "Нельзя задать размер с 0");
		return;
	}


	int line_count = this->input_grid->RowCount;
	
	for (int i = 0; i < line_count; i++) {
		this->input_grid->Rows->Remove(this->input_grid->Rows[0]);
		this->X_input->Rows->Remove(this->X_input->Rows[0]);
	}
	int column_count = this->input_grid->ColumnCount;
	for (int i = 0; i < column_count; i++) {
		this->input_grid->Columns->Remove(this->input_grid->Columns[0]);
	}
	
	
	for (int i = 0; i < m; i++) {
		this->input_grid->Columns->Add("", "");
	}
	
	this->input_grid->Rows->Add(n);
	this->X_input->Rows->Add(n);
	
	
}
\end{minted}

Функция очистки вывода
\begin{minted}[style=bw,
 linenos=true,
 breaklines=true,
 numbersep=5pt,
 tabsize=2,
 fontsize=\small,
 bgcolor=white]{cpp}
System::Void clearoutput() {
	int line_count = this->output_grid->RowCount;

	for (int i = 0; i < line_count; i++) {
		this->output_grid->Rows->Remove(this->output_grid->Rows[0]);

	}
	int column_count = this->output_grid->ColumnCount;
	for (int i = 0; i < column_count; i++) {
		this->output_grid->Columns->Remove(this->output_grid->Columns[0]);
	}
}
\end{minted}

